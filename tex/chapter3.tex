% !TeX root=../main.tex
\chapter{آشنایی با ابزار‌های توسعه}
در تمام ابزارهای ذکر شده در ادامه این متن حتما باید به ورژن هر کدام دقت شود، ورژن‌ها باید با یکدیگر همخوانی داشته باشند در غیر این صورت مشکلاتی در کامپایل و اجرای برنامه به وجود می‌آید که به راحتی قابل رفع کردن نیستند. در انجام این پروژه عدم همخوانی ورژن‌های مختلف ابزارها با یکدیگر باعث ایجاد مشکلات فراوانی شد، به همین دلیل ورژن مورد نیاز هر ابزار در توضیحات پروژه ذکر شده است.

% ------------ Section 2.1
\section{ابزارهای ساده}
\begin{itemize}
	\item \textbf{ویرایشگر}\\
	برای برنامه نویسی این قرارداد هوشمند از ویرایشگر VSCode با نصب
	پلاگین مربوط به Solidity
\LTRfootnote{https://marketplace.visualstudio.com/items?itemName=JuanBlanco.solidity}
استفاده شده است. این پلاگین با یافتن اشتباه‌ها پیش از کامپایل و راهنمایی در نوشتن کد قرارداد کمک شایانی به افزایش سرعت توسعه می‌کند.

	\item \textbf{ورژن‌کنترل}\\
	این پروژه از روز نخست به صورت متن‌باز توسعه یافته، برای توسعه یک پروژه به صورت متن‌باز اولین ابزار مورد نیاز یک برنامه ورژن کنترل است که نسخه‌های متفاوت و تغییر یافته کدها را به صورت مرتب نگهداری کند. برای این منظور از گیت‌هاب استفاده شده.

	\item \textbf{پکیج‌های Node و NPM}\\
از آنجایی که کدهای سالیدیتی در واقع جاوا‌اسکریپت هستند، به ابزارهای توسعه اپلیکیشن‌های جاوااسکریپت برای توسعه سالیدیتی نیاز است. ابزارهایی مانند Node برای کامپایل کردن برنامه‌های جاوااسکریپت و npm که مدیریت پکیج‌های جاوااسکریپتی که نصب می‌شود را به عهده دارد.

\end{itemize}


