% !TeX root=../main.tex
\chapter{پیاده‌سازی}

% ------------ Section 4.1
\section{نوشتن کد قرارداد}
در این بخش به بررسی مراحل و نحوه نوشتن کد قراردادهوشمند پرداخته می‌شود.

% ------------ Sub Section 4.1.1
\subsection{نیازمندی‌های قراردادهوشمند}
نیازمندی‌های اصلی کاپو به ترتیب زیر است.
\begin{itemize}
  \item
هر آدرس در شبکه بتواند یک داده‌ی متنی را به آسان‌ترین و کم هزینه‌ترین روش ممکن به یک توکن NFT تبدیل کند.
  \item
هر آدرس بتواند توکن‌های خود را به اشخاص دیگر انتقال دهد یا در بازارهای معاملات NFT بفروشد.
  \item
در صفحه اول وبسایت تعداد کل توکن‌های ساخته شده تا به حال و تعداد کل دارندگان توکن نمایش داده شود.
  \item
قابلیت‌های قراردادهوشمند تست شده باشد.
\end{itemize}

% ------------ Sub Section 4.1.2
\section{ارث‌بری}
با توجه به مزایای ذکر شده در مورد استانداردسازی قراردادهای هوشمند، انتخاب درستی است که برای پیاده‌سازی این کاربری از یکی از استانداردها استفاده شود. ارثبری از استانداردهای یک کتابخانه متن‌باز مزایای زیر را فراهم می‌کند.
\begin{itemize}
  \item
به دلیل وجود کدهای پایه به صورت آماده سرعت توسعه پروژه افزایش می‌یابد.
  \item
ارتباط دیگر پروژه‌ها با پروژه کاپو به راحتی انجام می‌شود.
  \item
امینت قرارداد و درستی آن حداقل در سطوح پایه‌ای تا حد خوبی تضمین شده است.
\end{itemize}

قراردادهوشمند کاپو از استاندارد ERC721 پیاده‌سازی شده در کتابخانه اپن‌زپلین
\LTRfootnote{
  \url{https://github.com/OpenZeppelin/openzeppelin-contracts}
}
ارث‌بری میکند که یکی از معروف ترین کتابخانه‌های پیاده کننده استانداردهای قرارداد هوشمند است.

% ------------ Sub Section 4.1.3
\subsection{توجه به هزینه تراکنش و نوع توابع}
در نوشتن یک قراردادهوشمند باید به نکات زیر توجه کنیم.
\begin{itemize}
  \item
میزان حافظه‌ای که اشغال می‌کنیم.
  \item
حجم بایت‌کد.
  \item
میزان عملیات هر متد، به خصوص متدهایی که مکررا مورد استفاده کاربر قرار می‌گیرند.
  \item
نوع هر متد، که مشخص می‌کند هر متد تا چه حد روی شبکه بلاکچین تغییر ایجاد می‌کند.
\end{itemize}

توجه نکردن به هریک از این موضوعات باعث می‌شود که قراردادهوشمند به اندازه کافی بهینه عمل نکند و کاربر وادار به پرداخت gasfee یا هزینه تراکنش بیشتر شود. یکی از مهمترین نکاتی که برای بهینه‌تر رفتار کردن قراردادهوشمند باید به آن توجه کنیم نوع هر متد است.

اگر متدی از نوع pure تعریف شود به این معنی است که به هیچ اطلاعاتی از شبکه بلاک‌چین نیاز ندارد و همه‌ی اطلاعاتی که لازم دارد را در اسکوپ
\LTRfootnote{Scope}
خودش دارد. اگر متدی از نوع view باشد به این معنی است که به اطلاعاتش روی شبکه بلاکچین نیاز دارد اما فقط می‌خواهد که آن‌ها را بخواند و نمیخواهد تغییری در آن‌ها ایجاد کند. این دو نوع متد نیازی به پرداخت کارمزد تراکنش توسط کاربر ندارند، اما اگر در تعریف متدی ذکر نشود که یکی از این دو نوع است، اینطور در نظر گرفته می‌شود که نیاز به بروزرسانی اطلاعاتش در شبکه بلاکچین دارد و از کاربری که آن را فراخوانی کرده است هزینه تراکنش دریافت می‌شود.


% ------------ Sub Section 4.1.4
\subsection{جزئیات فنی پیاده‌سازی}
مینت کردن در این قرارداد به آدرس‌های مشخص محدود نیست و همه می‌توانند توکن بسازند. بسیاری از قراردادها برای صرفه‌جویی در هزینه تراکنش کاربران اکثر اطلاعات مربوط به توکن‌ها را در قرارداد نگه نمیدارند و فقط داده‌های بسیار مهم توکن را در شبکه بلاکچین نگهداری می کنند. از آنجایی که کاپو یک قرارداد همه منظوره است و ممکن است استفاده‌های فراوانی داشته باشد، تصمیم‌گیری این مورد به عهده کاربر قرارداد گذاشته می‌شود.

در کاپو آیدی هر توکن از hash داده‌های توکن به دست می‌آید. این نحوه عملکرد چند مذیت ایجاد می‌کند. به این ترتیب هیچ دو توکنی نمی‌توانند داده‌های یکسان داشته باشند، زیرا در این صورت آیدی آن‌ها باید یکسان باشد و این امکان پذیر نیست زیرا آیدی توکن‌ها یکتاست. همچنین آیدی توکن‌ها دیگر ترتیبی نخواهند بود و ترتیب ساخت توکن‌ها مشخص نخواهد بود.

در یک قرارداد ERC721 استاندارد فقط آیدی توکن‌ها ذخیره می‌شود. در کاپو علاوه بر آیدی توکن‌ها یک map از آیدی توکن‌ها به داده‌ی آن‌ها با نام tokenDatas نیز نگهداری می‌شود. همچنین در کاپو map دیگری نیز از آدرس به لیست توکن‌های آن آدرس با نام ownerTokens نگهداری می‌شود. متغیر اول کمک می‌کند که با داشتن آیدی یک توکن به راحتی داده‌های آن توکن به دست آورده شوند. متغیر دوم نیز کمک می‌کند که به راحتی بتوان توکن‌های یک آدرس را به دست آورد. دومتغیر دیگر با نام‌های numberOfTokenHolders و numberOfMintedTokens نیز در کاپو نگه‌داشته می‌شوند که برای نمایش آمار استفاده از قرارداد در صفحه اصلی اپلیکیشن مورد استفاده قرار می‌گیرند.

متد mint به نحوی نوشته شده است که برای عموم قابل استفاده باشد. پس از محاسبه hash داده‌ی توکن از آن به عنوان آیدی توکن استفاده میکند، توکن را می‌سازد و متغیرهای tokenDatas و numberOfMintedTokens را بروزرسانی می‌کند.

\begin{figure}[ht]
\centerline{\includegraphics[width=12cm]{mint.png}}
\caption{پیاده‌سازی تابع mint}
\label{fig:mint}
\end{figure}

متد afterTokenTransfer از استاندارد ERC721 به نحوی بازنویسی
\LTRfootnote{Overwrite}
شده است که پس از هر انتقال توکن با بررسی آدرس‌های مبدا و مقصد، متغیر‌های numberOfTokenHolders و numberOfMintedTokens و ownerTokens را بروزرسانی کند.

\begin{figure}[ht]
\centerline{\includegraphics[width=12cm]{afterTokenTransfer.png}}
\caption{پیاده‌سازی تابع afterTokenTransfer}
\label{fig:afterTokenTransfer}
\end{figure}

متد جدیدی با نام getUserTokens نیز نوشته شده است که در استاندارد ERC721 به صورت پیش‌فرض وجود ندارد. این متد با گرفتن یک آدرس و استفاده از ownerTokens و tokenDatas دو خروجی برمی‌گرداند، لیستی از آیدی توکن‌های آدرس و لیستی از داده‌های توکن‌های آدرس.

\begin{figure}[ht]
\centerline{\includegraphics[width=12cm]{getUserTokens.png}}
\caption{پیاده‌سازی تابع getUserTokens}
\label{fig:getUserTokens}
\end{figure}

همچنین از آنجایی که سالیدیتی به طور پیش‌فرض امکان حذف یک داده از یک آرایه با داشتن مقدار آن را ندارد، عدم وجود این قابلیت هزینه‌بر بودن آن است، در سالیدیتی توسعه دهندگان به استفاده از map و دوری از array ها تشویق می‌شوند. اما برای نمایش نحوه ارث‌بری از دو یا چند قرارداد پدر، برای کاپو یک قرارداد به نام Helper نوشته شد که این قابلیت را فراهم می‌کند. کاپو علاوه بر
\lr{ERC721}
از قرارداد Helper نیز ارث‌بری می‌کند.

\begin{figure}[ht]
\centerline{\includegraphics[width=12cm]{Helper.png}}
\caption{پیاده‌سازی قرارداد Helper}
\label{fig:Helper}
\end{figure}

% ------------ Section 4.2
\section{نوشتن و اجرای تست‌ها}
پیش‌تر اشاره شد که از مزیت‌های ارث‌بری از کتابخانه‌های متن‌باز معروف این است که احتمال وجود خطا و مشکل امنیتی به شدت کمتر می‌شود. یکی از دلایل این مسئله این است که این کتابخانه‌ها پوشش تستی به شدت بالایی دارند. به همین دلیل می‌توان تا حدی به عملکرد قرارداد پدر اطمینان خاطر داشت و بیشتر روی تست کردن قابلیت‌های اضافه شده در قراردادهوشمند فرزند تمرکز داشت.

در کاپو برای هر عملکرد قرارداد تست نوشته شده است. یکی از ساده‌ترین تست‌های نوشته شده تست فرآیند ساخت یک توکن است که در آن پس از دیپلوی قرارداد با فراخوانی متد mint یک توکن ساخته می‌شود و سپس با فراخوانی متد balanceOf دارایی آدرس سازنده توکن بررسی می‌شود و انتظار می‌رود که پس از ساخت یک توکن، دارایی آدرس سازنده توکن یک باشد. این تست را می‌توان در تصویر زیر مشاهده کرد.

\begin{figure}[ht]
\centerline{\includegraphics[width=12cm]{mint-test.png}}
\caption{نمونه یکی از تست‌های قرارداد کاپو}
\label{fig:mint-test}
\end{figure}

پس از نوشته شدن تست‌ها می‌توان آن‌ها را با اجرای دستور
\lr{truffle test}
اجرا کرد. این دستور پس از اجرای تست‌ها نتیجه و زمان اجرای هر تست را به عنوان خروجی نمایش می‌دهد. نمونه اجرای این دستور را می‌توان در تصویر زیر مشاهده کرد.

\begin{figure}[ht]
\centerline{\includegraphics[width=12cm]{test-output.png}}
\caption{نمونه خروجی اجرای تست‌های قرارداد}
\label{fig:test-output}
\end{figure}

% ------------ Section 4.3
\section{دیپلوی قرارداد روی شبکه تستی Ropsten}
تا اینجا قراردادهوشمند نوشته و تست شده است، در این مرحله روی شبکه تستی Ropsten دیپلوی می‌شود. فرآیند دیپلوی شدن کاپو به کمک فریمورک ترافل قدم به قدم شرح داده می‌شود.

\subsection{یافتن آدرس یکی از نود‌های شبکه برای ارسال تراکنش دیپلوی قرارداد به آن}
آدرس نود‌های یک شبکه بلاکچین همه به صورت عمومی در دسترس هستند زیرا نودها باید بتوانند یکدیگر را ببینند. راه‌های زیادی برای به دست آوردن آدرس یک نود وجود دارد. یکی از آسان‌ترین راه‌های به دست آوردن آدرس یکی از نود‌های شبکه مراجعه به وبسایت ماینر است. برای این پروژه از وبسایت Moralis
\LTRfootnote{
  \url{https://moralis.io}
}
برای پیدا کردن آدرس نود شبکه استفاده شد.

\begin{figure}[ht]
\centerline{\includegraphics[width=12cm]{moralis.png}}
\caption{دریافت آدرس یکی از نود‌های شبکه از وبسایت Moralis}
\label{fig:moralis}
\end{figure}

\subsection{اضافه شدن اطلاعات شبکه مورد نظر به تنظیمات ترافل}
هنگامی که به کمک دستور
\lr{truffle init}
یک پروژه ترافل ساخته می‌شود، فایلی با نام truffle-config.js ساخته می‌شود. تنظیمات مربوط به ترافل در این فایل نوشته شده‌است. برای این که ترافل شبکه مورد نظر را بشناسد باید اطلاعات آن شبکه در این فایل نوشته و شبکه‌ی جدیدی تعریف شود. برای تعریف شبکه از آدرسی که در گام قبل به دست آمد استفاده می‌شود و مانند تصویر زیر شبکه‌ی جدیدی تعریف می‌شود.

\begin{figure}[ht]
\centerline{\includegraphics[width=12cm]{network-config.png}}
\caption{اضافه کردن شبکه Ropsten به شبکه‌های ترافل}
\label{fig:network-config}
\end{figure}


\subsection{آماده شدن mnemonics}
برای انجام این پروژه به کمک دستور
\lr{npm mnemonics}
یک آدرس تستی ساخته می‌شود. این دستور، mnemonics متناسب با این آدرس را به عنوان خروجی می‌دهد. دقت کنید که برای دیپلوی روی
\gls{Mainnet}
حتما باید از mnemonics مربوط به یک کیف پول واقعی استفاده شود و اطلاعات ان در اختیار کسی قرار نگیرد.

\begin{figure}[ht]
\centerline{\includegraphics[width=12cm]{mnemonics.png}}
\caption{ایجاد mnemonics تستی}
\label{fig:mnemonics}
\end{figure}


\subsection{استفاده از کیف پول ایجاد شده در تنظیمات ترافل}
ترافل برای این که بتواند از کیف‌پول برای انجام تراکنش‌ها استفاده کند باید به mnemonics یا کلید خصوصی آن دسترسی داشته باشد. به این منظور فایلی با نام secrets.json در دایرکتوری اصلی برنامه ساخته می‌شود و mnemonics کیف پول به شکل زیر در آن قرار داده می‌شود.

\begin{figure}[ht]
\centerline{\includegraphics[width=12cm]{mnemonics-in-secrets.png}}
\caption{قراردادن mnemonics در فایل secrets.json}
\label{fig:mnemonics-in-secrets}
\end{figure}

سپس در تنظیمات ترافل باید ذکر شود که می‌تواند آدرس کیف‌پول را در این آدرس پیدا کند.

\begin{figure}[ht]
\centerline{\includegraphics[width=12cm]{secrets-in-config.png}}
\caption{معرفی فایل secrets.json در تنظیمات ترافل}
\label{fig:secrets-in-config}
\end{figure}


\subsection{نصب کیف‌پول hdwallet}
ترافل برای استفاده از mnemonics کیف پول ما نیاز به نصب پکیج hdwallet-provider دارد، این پکیج کاربری‌های یک کیف پول دیجیتال از جمله امضا و ارسال تراکنش بر روی شبکه بلاکچین را در اختیار ترافل قرار می‌دهد. این پکیج با اجرای دستور
\lr{npm install –save-dev @truffle/hdwallet-provider}
نصب می‌شود.  پس از نصب کیف پول در تنظیمات ترافل در فایل truffle-config.js ذکر می‌شود که از این کیف‌پول استفاده شود.

\begin{figure}[ht]
\centerline{\includegraphics[width=12cm]{wallet-in-config.png}}
\caption{استفاده از کیف‌پول hdwallet در تنظیمات ترافل}
\label{fig:wallet-in-config}
\end{figure}


\subsection{انتخاب شبکه اضافه شده}
حال هنگام ورود به خط فرمان ترافل مانند تصویر زیر شبکه مورد نظر انتخاب می‌شود.

\begin{figure}[ht]
\centerline{\includegraphics[width=12cm]{truffle-console.png}}
\caption{ورود به خط فرمان ترافل با انتخاب شبکه Ropsten}
\label{fig:truffle-console}
\end{figure}


\subsection{بررسی آدرس کیف‌پول و موجودی آن}
برای دیپلوی یک قرارداد هوشمند باید آدرس دیپلوی کننده آن بتواند هزینه تراکنش دیپلوی را پرداخت کند. در صورتی که دیپلوی بر روی یک شبکه تستی انجام می‌شود باید با استفاده از یک faucet روی شبکه تستی به میزان کافی پول تستی دریافت شود.

برای دریافت آدرس‌های کیف‌پول از دستور زیر در خط فرمان ترافل استفاده می شود.

\begin{figure}[ht]
\centerline{\includegraphics[width=12cm]{get-addresses.png}}
\caption{دریافت آدرس‌های کیف‌پول در خط فرمان ترافل}
\label{fig:get-addresses}
\end{figure}

برای دریافت مانده حساب آدرس، دستور زیر در خط فرمان ترافل اجرا می‌شود.

\begin{figure}[ht]
\centerline{\includegraphics[width=12cm]{get-wallet-balance.png}}
\caption{دریافت موجودی کیف‌پول در خط فرمان ترافل}
\label{fig:get-wallet-balance}
\end{figure}


\subsection{دیپلوی قراردادهوشمند روی شبکه بلاکچین}
پس از اطمینان از توانایی پرداخت کارمزد تراکنش با استفاده از دستور migrate در خط فرمان ترافل قراردادهوشمند روی شبکه بلاکچین دیپلوی می‌شود.


\subsection{اطمینان از صحت دیپلوی قراردادهوشمند}
س از اتمام دیپلوی قرارداد هوشمند برای اطمینان از به درستی انجام شدن فرآیند دیپلوی قرارداد، می‌توان از
\glspl{Block Explorer}
بلاکچین استفاده کرد. برای مثال قرارداد هوشمند کاپو بر روی شبکه Ropsten دیپلوی شده است، که با رفتن به وبسایت اتراسکن
\LTRfootnote{
\url{https://etherscan.io}
}
و قراردادن آن روی شبکه Ropsten می‌توان قرارداد دیپلوی شده و تراکنش‌های آن را مشاهده کرد.


\begin{figure}[ht]
\centerline{\includegraphics[width=12cm]{etherscan.png}}
\caption{مشاهده قراداد کاپو در Etherscan روی شبکه Ropsten}
\label{fig:etherscan}
\end{figure}


% ------------ Section 4.4
