% !TeX root=../main.tex
\chapter{دست‌آوردها، پیشنهاد‌ها، محدودیت‌ها}

\section{دست‌آوردها}

\subsection{یادگیری}
در طی انجام این پروژه با ابزارها، فریمورک‌ها، کتابخانه‌ها و استانداردهای توسعه قراردادهای هوشمند آشنا شدیم. آموختیم که فریمورک ترافل چه ابزارهایی را در اختیار توسعه دهنده قرار می‌دهد. چگونه می‌توان یک شبکه لوکال برای توسعه ایجاد کرد، قرارداد هوشمند را بر روی آن دیپلوی کرد و فرانت‌اند و کیف پول را به آن متصل کرد.

آموختیم که چگونه می‌توانیم برای پیاده‌سازی قراردادهای هوشمند از استانداردهای موجود استفاده کنیم، برای آن‌ها تست بنویسیم و به کمک فریمورک ترافل این تست‌ها را اجرا کنیم. آموختیم که چگونه پس از اتمام فرآیند توسعه قراردادهوشمند را بر روی شبکه تستی دیپلوی کنیم. همچنین فرانت‌اند اپلیکیشن به کمک صفحات گیت‌هاب دیپلوی و به قرارداد هوشمند روی شبکه تست متصل شد.

\subsection{پلتفرم ایجاد شده}
قرارداد هوشمند نوشته شده در این پروژه، کاپو، با عملکرد کامل بر بروی شبکه تستی Ropsten دیپلوی شد و امکانات لازم برای دسترسی عموم مردم به روشی آسان و ارزان به توکن‌های تعویض ناپذیر را فراهم می‌کند. کاربران می‌توانند در صفحه اصلی این اپلکیشن تعداد توکن‌های ساخته شده و تعداد آدرس‌های دارای توکن را مشاهده کنند. سپس با متصل کردن کیف‌پولشان به اپلیکیشن می‌توانند توکن بسازند، دارایی‌هایشان را مشاهده کنند و توکن‌هایشان رو به دیگران ارسال کنند.

\subsection{ساخت محیط توسعه سریع و خودکار}
سپس آموختیم که چگونه اجرای تست‌های قرارداد هوشمند را داکرایز و به صورت خودکار در پایپ‌لاین پروژه اجرا کنیم. برای انجام این کار یک داکر ایمیج ترافل نوشته شد، کد آن به صورت متن‌باز بر روی گیت‌هاب بارگزاری و ایمیج آن به داکرهاب اضافه شد. سپس فرانت‌اند اپلیکیشن داکرایز شد و از env های داکر برای فرستادن توکن گیت‌هاب از پایپلاین به کانتینر استفاده شد و در نتیجه فرانت‌اند اپلیکیشن به صورت خودکار در پایپلاین پروژه روی صفحات گیت‌هاب بارگزاری می شود.

در نتیجه انجام این کارها یک مسیر راحت و سریع برای توسعه یک قرارداد هوشمند به همراه فرانت‌اند ایجاد شد که تست‌ها و فرایند دیپلوی همه به صورت خودکار در آن اجرا می‌شوند.



\section{پیشنهادها}

\subsection{استفاده از استانداردها}

\subsection{استفاده از ERC1155 به جای ERC721}

\subsection{ساخت محیط توسعه از شروع کار}



\section{محدودیت‌ها}

\subsection{استفاده از ERC721}

\subsection{}

\subsection{}

