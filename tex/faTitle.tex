% !TeX root=../main.tex
% در این فایل، عنوان پایان‌نامه، مشخصات خود، متن تقدیمی‌، ستایش، سپاس‌گزاری و چکیده پایان‌نامه را به فارسی، وارد کنید.
% توجه داشته باشید که جدول حاوی مشخصات پروژه/پایان‌نامه/رساله و همچنین، مشخصات داخل آن، به طور خودکار، درج می‌شود.
%%%%%%%%%%%%%%%%%%%%%%%%%%%%%%%%%%%%
% دانشگاه خود را وارد کنید
\university{دانشگاه تهران}
% پردیس دانشگاهی خود را اگر نیاز است وارد کنید (مثال: فنی، علوم پایه، علوم انسانی و ...)
\college{پردیس دانشکده‌های فنی}
% دانشکده، آموزشکده و یا پژوهشکده  خود را وارد کنید
\faculty{دانشکده مهندسی برق و کامپیوتر}
% گروه آموزشی خود را وارد کنید (در صورت نیاز)
\department{گروه نرم‌افزار}
% رشته تحصیلی خود را وارد کنید
\subject{مهندسی کامپیوتر}
% گرایش خود را وارد کنید
\field{نرم‌افزار}
% عنوان پایان‌نامه را وارد کنید
\title{کاپو، پلتفرم ساخت و انتقال توکن‌های داده‌ای}
% نام استاد(ان) راهنما را وارد کنید
\firstsupervisor{دکتر احسان خامس‌پناه}
\firstsupervisorrank{استاد}
% \secondsupervisor{دکتر راهنمای دوم}
% \secondsupervisorrank{استادیار}
% نام استاد(دان) مشاور را وارد کنید. چنانچه استاد مشاور ندارید، دستورات پایین را غیرفعال کنید.
% \firstadvisor{دکتر مشاور اول}
\firstadvisorrank{استادیار}
%\secondadvisor{دکتر مشاور دوم}
% نام داوران داخلی و خارجی خود را وارد نمایید.
\internaljudge{دکتر داور داخلی}
\internaljudgerank{دانشیار}
\externaljudge{دکتر داور خارجی}
\externaljudgerank{دانشیار}
\externaljudgeuniversity{دانشگاه داور خارجی}
% نام نماینده کمیته تحصیلات تکمیلی در دانشکده \ گروه
\graduatedeputy{دکتر نماینده}
\graduatedeputyrank{دانشیار}
% نام دانشجو را وارد کنید
\name{امین}
% نام خانوادگی دانشجو را وارد کنید
\surname{بشیری}
% شماره دانشجویی دانشجو را وارد کنید
\studentID{810196425}
% تاریخ پایان‌نامه را وارد کنید
\thesisdate{خرداد ۱۴۰۱}
% به صورت پیش‌فرض برای پایان‌نامه‌های کارشناسی تا دکترا به ترتیب از عبارات «پروژه»، «پایان‌نامه» و «رساله» استفاده می‌شود؛ اگر  نمی‌پسندید هر عنوانی را که مایلید در دستور زیر قرار داده و آنرا از حالت توضیح خارج کنید.
%\projectLabel{پایان‌نامه}

% به صورت پیش‌فرض برای عناوین مقاطع تحصیلی کارشناسی تا دکترا به ترتیب از عبارت «کارشناسی»، «کارشناسی ارشد» و «دکتری» استفاده می‌شود؛ اگر نمی‌پسندید هر عنوانی را که مایلید در دستور زیر قرار داده و آنرا از حالت توضیح خارج کنید.
%\degree{}
%%%%%%%%%%%%%%%%%%%%%%%%%%%%%%%%%%%%%%%%%%%%%%%%%%%%
%% پایان‌نامه خود را تقدیم کنید! %%
% \dedication
% {
% {\Large تقدیم به:}\\
% \begin{flushleft}{
% 	\huge
% 	همسر و فرزندانم\\
% 	\vspace{7mm}
% 	و\\
% 	\vspace{7mm}
% 	پدر و مادرم
% }
% \end{flushleft}
% }
%% متن قدردانی %%
%% ترجیحا با توجه به ذوق و سلیقه خود متن قدردانی را تغییر دهید.
% \acknowledgement{
% سپاس خداوندگار حکیم را که با لطف بی‌کران خود، آدمی را به زیور عقل آراست.

% در آغاز وظیفه‌  خود  می‌دانم از زحمات بی‌دریغ اساتید  راهنمای خود،  جناب آقای دکتر ... و ...، صمیمانه تشکر و  قدردانی کنم که در طول انجام این پایان‌نامه با نهایت صبوری همواره راهنما و مشوق من بودند و قطعاً بدون راهنمایی‌های ارزنده‌ ایشان، این مجموعه به انجام نمی‌رسید.

% از جناب آقای دکتر ... که  زحمت مشاوره‌، بازبینی و تصحیح این پایان‌نامه را تقبل فرمودند کمال امتنان را دارم.

% %از همکاری و مساعدت‌های دکتر ... مسئول تحصیلات تکمیلی و سایر کارکنان دانشکده بویژه سرکار خانم ... کمال تشکر را دارم.

% با سپاس بی‌دریغ خدمت دوستان گران‌مایه‌ام، خانم‌ها ... و آقایان ... در آزمایشگاه ...، که با همفکری مرا صمیمانه و مشفقانه یاری داده‌اند.

% و در پایان، بوسه می‌زنم بر دستان خداوندگاران مهر و مهربانی، پدر و مادر عزیزم و بعد از خدا، ستایش می‌کنم وجود مقدس‌شان را و تشکر می‌کنم از خانواده عزیزم به پاس عاطفه سرشار و گرمای امیدبخش وجودشان، که بهترین پشتیبان من بودند.
% }
%%%%%%%%%%%%%%%%%%%%%%%%%%%%%%%%%%%%
%چکیده پایان‌نامه را وارد کنید
\fa-abstract{
\gls{Decentralized Web}
یا
\lr{Web3}
 به عنوان مهمترین تغییر بعد از به وجود آمدن
\gls{Word Wide Web}
 در نظر گرفته می‌شود. با به وجود آمدن
\glspl{CryptoCurrency}
و
\glspl{Consensus Algorithms}
کارآمد، و فراهم شدن زمینه اجرای برنامه‌ها و انجام تراکنش‌های مالی به صورت غیر متمرکز،
عصر اینترنت غیرمتمرکز فرا رسیده است.

در این میان یکی از اصلی‌ترین مزایای 
\glspl{Dapp}
مالکیت واقعی دارایی است.
به این معنی که یک شخص یا یک نهاد نمی‌تواند دارایی‌های کس دیگری را مسدود یا مصادره کند.
از طرفی مالیکت‌های معنوی نیز به صورت واضح و شفاف می‌توانند مشخص شوند.
برای مثال یک هنرمند به وضوح صاحب اثرش است و هرچند که دیگر افراد می‌توانند اثر او را کپی کنند
اما همیشه مشخص است که صاحب اصلی اثر کیست.

به این ترتیب 
\glspl{Non-fungible token}
با قابلیت‌های مالکیت بسیار زیادی که فراهم می‌کنند مورد استقبال فراوان مردم واقع شدند.
قابلیت‌هایی مانند ساخت، نگهداری، فروش و انتقال فوق‌العاده راحت و سریع نیز
در این سرعت فراگیری تاثیر بسزایی داشته‌اند.
برای بازاری به این تازگی و وسعت، تکنولوژی‌ها، استاندارد‌ها و پلتفرم‌های زیادی ساخته شده‌اند
و همچنان نیز در حال توسعه هستند.

در این پروژه سعی بر ساخت پلتفرمی داریم که هر شخص یا شرکتی بتواند با عضویت در آن،
به آسان‌ترین روش ممکن توکن‌های تعویض‌ناپذیر‌ بسازد و به دیگران انتقال دهد.
کاربرد‌های این پلتفرم ساده بی‌شمار است.
دارایی‌هایی مانند بلیت سینما، ژتون‌های غذا، وقت گرفتن از دکتر، قراردادها و ... همه
می‌توانند به آسانی در این پلتفرم به توکن تعویض‌ناپذیر تبدیل شوند،
به دیگران انتقال یابند و در بازار خرید و فروش شوند.

گرچه می‌توان کاربردهای فراوانی را برای این پلتفرم در نظر داشت
اما همچنان یکی از اهداف انجام این پروژه آشنایی با نحوه ساخت، تست و بارگذاری یک قرارداد هوشمند،
ساخت واسط‌کاربری، اتصال آن به قرارداد هوشمند
و همچنین شناخت استاندارد‌های معروف قرارداد‌های توکن‌های تعویض‌ناپذیر مانند
\lr{ERC721}
و
\lr{ERC1155}
است.

لازم به ذکر است که تمامی کدهای کاپو به صورت
\gls{Open Source}
در 
گیت‌هاب
\LTRfootnote{\url{https://github.com/bshramin/cappu}}
قابل دسترس برای عموم هستند.
}
% کلمات کلیدی پایان‌نامه را وارد کنید
\keywords{
کاپو،
توکن،
داده،
سالیدیتی،
قرارداد هوشمند،
توکن غیرقابل تعویض،
ترافل
}
% انتهای وارد کردن فیلد‌ها
%%%%%%%%%%%%%%%%%%%%%%%%%%%%%%%%%%%%%%%%%%%%%%%%%%%%%%
