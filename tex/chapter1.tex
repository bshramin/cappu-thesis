% !TeX root=../main.tex

\chapter{مقدمه و بیان مسئله}
% دستور زیر باعث عدم‌نمایش شماره صفحه در اولین صفحهٔ این فصل می‌شود.
%\thispagestyle{empty}

% ------------ Section 1.1 ------------
\section{مقدمه}
در یک دهه اخیر محبوبیت رمزارز‌ها در میان مردم به شدت افزایش یافته است.
رمزارزها توکن هایی تعویض‌پذیر هستند، به این معنی که تفاوتی میان دو توکن یک
\gls{CryptoCurrency}
وجود ندارد، مانند
\gls{Fiat Money}
که ارزش یک هزار تومانی با یک هزار تومانی دیگر تفاوتی ندارد.

اما در دنیای واقعی تنها مالکیت پول نیست که اهمیت دارد،
بلکه یک فرد می‌تواند خودرو، خانه، بلیت هواپیما و دیگر دارایی‌هایی داشته باشد که یکتا هستند و
با هیچ دارایی دیگری دقیقا یکسان نیستند.
مثلا یک بلیت هواپیما برای تاریخ و ساعتی خاص برای شماره پروازی خاص از یک مبدا مشخص به یک مقصد مشخص است و
شماره صندلی یکتایی نیز دارد.
پس هیچ دو بلیت هواپیمایی دقیقا یکسان نیستند،
بر خلاف دو بیتکوین که کاملا یکسان هستند، ارزش برابری دارند، و تعویض‌پذیر هستند.

کاربردهای توکن‌های تعویض ناپذیر بیشمار است و
در حال حاضر فقط قسمت اندکی از کاربردهایی که می‌توانند داشته باشند را پاسخ گفته‌اند.
در این پروژه یک
\gls{Platform}
ساخته می‌شود که
\gls{Mint}
و انتقال توکن‌های تعویض ناپذیر را برای عموم در دسترس‌تر و آسان‌تر می‌کند.
همچنین یکی از اهداف انجام این پروژه آشنایی با تکنولوژی‌ها، استاندارد‌ها و فرایند‌های توسعه این توکن‌هاست.


% ------------ Section 1.2
\section{شرح مسئله و روش انجام آن}
پروژه تعریف شده توسعه یک پلتفرم برای ساخت و انتقال توکن‌های تعویض ناپذیر به آسان‌ترین روش ممکن است.
به نحوی که برای هر کسی به راحتی در دسترس باشد.
نکته‌ی قابل توجه‌ این است که در مسیر انجام این پروژه با تکنولوژی‌های موجود در این زمینه، 
\glspl{Framework}
،استاندارد‌ها و فرایند تست و بارگذاری آشنا شویم.

برای انجام این مراحل در قدم اول نحوه توسعه اپلیکیشن‌های غیرمتمرکز و
برتری‌های نوشتن پروژه به صورت متن‌باز ذکر می‌شود،
سپس چارچوب‌ها و ابزار‌هایی که برای ساخت یک اپلیکیشن غیرمتمرکز به توسعه دهنده کمک می‌کنند معرفی شده
و نحوه استفاده از آن‌ها شرح داده می‌شود.

سپس فرایند توسعه آغاز می‌شود،
استاندارد‌های موجود برای نوشتن یک قرارداد برای توکن‌های تعویض ناپذیر شرح داده می‌شود و
کاپو تا جای ممکن مطابق آن‌ها توسعه می‌یابد.
برای قرارداد هوشمند نوشته شده تست می‌نویسیم و آن را روی
\gls{Testnet}
انتشار می‌دهیم.
در گام بعد برای پلتفرم،
واسط کاربری ساده‌ای نوشته می‌شود که با قرارداد هوشمند و
همچنین کیف پول دیجیتال کاربر ارتباط برقرار می‌کند و سپس به کمک
\gls{Github Pages}
بارگذاری می‌شود تا در دسترس عموم کاربرها قرار بگیرد.

برای
\gls{Dockerize}
کردن تست‌های
\gls{Smart Contract}
یک
\gls{Docker Image}
\gls{Truffle}
نوشته می‌شود.
در قدم بعد هر دو بخش واسط کاربری و قرارداد هوشمند داکرایز می‌شوند
و فرایند اجرای تست‌های قرارداد هوشمند
و بارگذاری شدن واسط کاربری به صورت خودکار به کمک پایپلاین‌های گیت‌هاب پیاده‌سازی می‌شود.


% ------------ Section 1.3
\section{اهداف کلی تحقیق}
\subsection{گسترش کاربرد‌های توکن‌های تعویض ناپذیر}
این توکن‌ها در همین مدت کوتاهی که به وجود آمده‌اند کاربردهای فراوانی را پوشش داده‌اند.
اما همچنان قسمت بزرگی از این کاربردها صرفا ثبت مالکیت آثار هنری دیجیتال است.
درحالی که توکن‌های داده‌ای می‌توانند وسعت بسیار عظیم‌تری از کاربردها را پوشش دهند.
از کاربردهای روزانه مانند بلیت سینما و هواپیما، تا مالکیت هر نوع دارایی واقعی یا مجازی.

با توجه به نحوه کار اکثر قراردادهای توکن‌های تعویض ناپذیر،
معمولا فقط مالک قرارداد می‌تواند توکن ایجاد کند،
یا در قرارداد برای ایجاد توکن شرط‌هایی مانند حداکثر تعداد ممکن گذاشته می‌شود.
این موضوع به این معنی است که اگر شخصی بخواهد خودش توکن‌هایی ایجاد کند
و به دیگران انتقال دهد احتمالا مجبور است که قرارداد هوشمند خودش را بنویسد و بارگذاری کند.
این فرآیند نیاز به دانش فنی، آشنایی کامل با این زمینه و پرداخت هزینه‌های بارگذاری قرارداد روی شبکه بلاکچین دارد.

کاپو به هر آدرسی اجازه می‌دهد که به راحت‌ترین حالت ممکن
و به هر تعداد که مورد نیاز است توکن تعویض ناپذیر روی این قرارداد ایجاد کند.
به این ترتیب استفاده از کاپو برای عموم مردم آسان‌تر، ارزان‌تر و در دسترس‌تر است.


\subsection{ساخت
\gls{Development Flow}
قرارداد هوشمند}
یکی از اهداف انجام این پروژه این است که پس از آشنایی با ایزارهای موجود،
یک ساختار برای روند توسعه قرارداد هوشمند ایجاد شود.
این ساختار به نحوی خواهد بود که توسعه قرارداد سریع‌تر و با اطمینان خاطر بیشتری انجام شود.
در ساخت روند توسعه بیشتر تاکید بر راحتی اضافه کردن تست‌ها و اجرای خودکار تست‌ها،
آسان و خودکار بودن روند دیپلوی و اتصال بی دردسر فرانت‌اند به قرارداد هوشمند است.

\subsection{یادگیری}
هدف دیگر انجام این پروژه یادگیری است.
با توجه به رشد سریع و تازگی استفاده از تکنولوژی‌های بلاکچین و توکن‌های تعویض ناپذیر،
با وجود تلاش برای ایجاد منابع یادگیری مناسب همچنان فضاهای خالی، کمبودها و نیازمندی‌هایی وجود دارد که باید پاسخ گفته شوند.
در طی انجام این پروژه با ابزارها، کتابخانه‌ها، چارچوب‌ها و استانداردهای نوشتن قراردادهای هوشمند آشنا می‌شویم،
می‌آموزیم که هر یک چطور کار میکنند و چگونه می‌توانند به توسعه دهنده کمک کنند.

% ------------ Section 1.4
\section{ساختار پایان‌نامه}
پس از این مقدمه، در فصل ۲ مفاهیم اولیه توسعه اپلیکیشن بر بستر بلاک‌چین، کاربردها، مفاهیم و استاندارد‌ها توضیح داده می‌شود.
در فصل ۳ ابزار‌های توسعه قرارداد‌های هوشمند معرفی می‌شوند،
مزایا و معایب هر یک بیان می‌شود و نحوه استفاده از آن‌ها توضیح داده می‌شود.
در فصل ۴ روند پیاده‌سازی شرح داده می‌شود. بررسی می‌شود که در هر مرحله از پیاده‌سازی چه کارهایی به چه ترتیبی انجام شده است.
در فصل پنجم نیز نتایج توضیح داده می‌شوند و جمع‌بندی صورت میگیرد.
