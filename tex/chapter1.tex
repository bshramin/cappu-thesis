% !TeX root=../main.tex

\chapter{مقدمه و بیان مسئله}
% دستور زیر باعث عدم‌نمایش شماره صفحه در اولین صفحهٔ این فصل می‌شود.
%\thispagestyle{empty}

% ------------ Section 2
\section{مقدمه}
در یک دهه اخیر محبوبیت رمزارز‌ها در میان مردم به شدت افزایش داشته است. رمزارزها توکن هایی تعویض پذیر هستند به این معنی که تفاوتی میان دو توکن یک
\gls{CryptoCurrency}
وجود ندارد، مانند پول فیزیکی که ارزش یک هزار تومانی با یک هزار تومانی دیگر تفاوتی ندارد.
اما در دنیای واقعی تنها مالکیت پول وجود ندارد، بلکه یک فرد میتواند خودرو، خانه، بلیت هواپیمای و دیگر دارایی‌هایی داشته باشد که یکتا هستند و با هیچ دارایی دیگری دقیقا یکسان نیستند. مثلا یک بلیت هواپیما برای یک تاریخ و ساعت خاص برای یک شماره پرواز خاص از یک مبدا مشخص به یک مقصد مشخص است و شماره صندلی یکتایی نیز دارد. پس هیچ دو بلیت هواپیمایی دقیقا یکسان نیستند، بر خلاف دو بیتکوین که کاملا یکسان هستند، ارزش برابری دارند، و تعویض پذیر هستند.
کاربردهای
\glspl{Non-fungible token}
بیشمار است و در حال حاضر فقط قسمت اندکی از کاربردهایی که میتوانند داشته باشند را پاسخ گفته‌اند. در این پروژه پلتفرمی می‌سازیم که ساخت و انتقال
\gls{Non-fungible token}
را برای عموم در دسترس‌تر و آسان‌تر کند. همچنین یکی از اهدف انجام این پروژه آشنایی با تکنولوژی‌ها، استاندارد‌ها و فرایند‌های این توکن‌هاست.

% ------------ Section 2
\section{شرح مسئله و روش انجام آن}
پروژه تعریف شده ساخت یک
\gls{Platform}
برای ساخت و انتقال
\gls{Non-fungible token}
به آسان‌ترین روش ممکن است، به نحوی که برای هر کسی به راحتی در دسترس باشد.

نکته‌ی قابل توجه‌ این است که در مسیر انجام این پروژه با تکنولوژی‌های موجود در این زمینه، فریمورک‌ها، استاندارد‌ها و فرایند تست و دیپلوی آشنا شویم.

برای انجام این مراحل در قدم اول با نحوه
\gls{Develop}
\glspl{Dapp}
و برتری‌های نوشتن پروژه به صورت
\gls{Open Source}
آشنا میشویم، سپس فریمورک‌ها و ابزار‌هایی که برای ساخت یک
\glspl{Dapp}
به ما کمک می‌کنند را می‌شناسیم و تصمیم میگیریم که کدام‌ یک را مورد استفاده قرار دهیم.

سپس فرایند
\gls{Develop}
را آغاز میکنیم، استاندارد‌های موجود برای نوشتن یک
\gls{Smart Contract}
برای
\glspl{Non-fungible token}
را می‌شناسیم و تا جای ممکن مطابق آن‌ها پیش میرویم. برای
\gls{Smart Contract}
نوشته شده تست می‌نویسیم و آن را روی
\gls{Testnet}
انتشار می‌دهیم.

در گام بعد برای
\gls{Platform}
، فرانت‌اند ساده‌ای می‌نویسیم که با
\gls{Smart Contract}
و همچنین
\gls{Wallet}
کاربر ارتباط برقرار کند و سپس به کمک
\gls{Github Pages}
آن‌ را دیپلوی میکنیم.

برای
\gls{Dockerize}
کردن
\gls{Smart Contract}
یک
\gls{Docker Image}
\gls{Truffle}
می‌نویسیم و در قدم بعد هر دو بخش فرانت و
\gls{Smart Contract}
\gls{Dockerize}
می‌شوند و فرایند اجرای تست‌های
\gls{Smart Contract}
و دیپلوی شدن فرانت به صورت
\gls{Automatic}
به کمک پایپلاین‌های گیت‌هاب پیاده‌سازی می‌شود.
