% !TeX root=../main.tex

\chapter{مقدمه و بیان مسئله}
% دستور زیر باعث عدم‌نمایش شماره صفحه در اولین صفحهٔ این فصل می‌شود.
%\thispagestyle{empty}

% ------------ Section 1.1 ------------
\section{مقدمه}
در یک دهه اخیر محبوبیت رمزارز‌ها در میان مردم به شدت افزایش داشته است. رمزارزها توکن هایی تعویض پذیر هستند به این معنی که تفاوتی میان دو توکن یک
\gls{CryptoCurrency}
وجود ندارد، مانند
\gls{Fiat Money}
که ارزش یک هزار تومانی با یک هزار تومانی دیگر تفاوتی ندارد.

اما در دنیای واقعی تنها مالکیت پول نیست که اهمیت دارد، بلکه یک فرد میتواند خودرو، خانه، بلیت هواپیما و دیگر دارایی‌هایی داشته باشد که یکتا هستند و با هیچ دارایی دیگری دقیقا یکسان نیستند. مثلا یک بلیت هواپیما برای تاریخ و ساعتی خاص برای شماره پروازی خاص از یک مبدا مشخص به یک مقصد مشخص است و شماره صندلی یکتایی نیز دارد. پس هیچ دو بلیت هواپیمایی دقیقا یکسان نیستند، بر خلاف دو بیتکوین که کاملا یکسان هستند، ارزش برابری دارند، و تعویض پذیر هستند.

کاربردهای توکن‌های تعویض ناپذیر بیشمار است و در حال حاضر فقط قسمت اندکی از کاربردهایی که میتوانند داشته باشند را پاسخ گفته‌اند. در این پروژه یک
\gls{Platform}
می‌سازیم که ساخت و انتقال توکن‌های تعویض ناپذیر را برای عموم در دسترس‌تر و آسان‌تر می‌کند. همچنین یکی از اهدف انجام این پروژه آشنایی با تکنولوژی‌ها، استاندارد‌ها و فرایند‌های توسعه این توکن‌هاست.


% ------------ Section 1.2
\section{شرح مسئله و روش انجام آن}
پروژه تعریف شده توسعه یک پلتفرم برای
\gls{Mint}
و انتقال توکن‌های تعویض ناپذیر به آسان‌ترین روش ممکن است، به نحوی که برای هر کسی به راحتی در دسترس باشد. نکته‌ی قابل توجه‌ این است که در مسیر انجام این پروژه با تکنولوژی‌های موجود در این زمینه، فریمورک‌ها، استاندارد‌ها و فرایند تست و دیپلوی آشنا شویم.

برای انجام این مراحل در قدم اول نحوه توسعه اپلیکیشن‌های غیرمتمرکز و برتری‌های نوشتن پروژه به صورت متن‌باز ذکر می‌شود، سپس فریمورک‌ها و ابزار‌هایی که برای ساخت یک اپلیکیشن غیرمتمرکز به توسعه دهنده کمک می‌کنند معرفی می‌شوند و نحوه استفاده از آن‌ها شرح داده می‌شود.

سپس فرایند توسعه آغاز می‌شود، استاندارد‌های موجود برای نوشتن یک قرارداد برای توکن‌های تعویض ناپذیر شرح داده می‌شود و کاپو تا جای ممکن مطابق آن‌ها توسعه می‌یابد. برای قرارداد هوشمند نوشته شده تست می‌نویسیم و آن را روی
\gls{Testnet}
انتشار می‌دهیم.
در گام بعد برای پلتفرم، فرانت‌اند ساده‌ای نوشته می‌شود که با قرارداد هوشمند و همچنین کیف پول دیجیتال کاربر ارتباط برقرار می‌کند و سپس به کمک
\gls{Github Pages}
دیپلوی می‌شود تا در دسترس عموم کاربرها قرار بگیرد.

برای
\gls{Dockerize}
کردن تست‌های
\gls{Smart Contract}
یک
\gls{Docker Image}
\gls{Truffle}
نوشته می‌شود. در قدم بعد هر دو بخش فرانت و قرارداد هوشمند داکرایز می‌شوند و فرایند اجرای تست‌های قرارداد هوشمند و دیپلوی شدن فرانت به صورت خودکار به کمک پایپلاین‌های گیت‌هاب پیاده‌سازی می‌شود.


% ------------ Section 1.3
\section{اهداف کلی تحقیق}
