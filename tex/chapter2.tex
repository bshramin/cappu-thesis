% !TeX root=../main.tex
\chapter{مفاهیم اولیه و پیش‌زمینه}
% ------------ Section 2.1
\section{دلایل و برتری‌های متن‌باز بودن قرارداد‌های هوشمند}
دلایل زیادی برای متن‌باز نوشتن قراردادهای هوشمند وجود دارد، در ادامه تعدادی از این دلایل توضیح داده می‌شود.

دلیل اول، بلاکچین‌ها
\gls{Confidentiality}
ندارند.
همه‌ی نُد‌های شبکه برای اجرای کد قرارداد هوشمند باید حداقل به
\glspl{bytecode}ی
قرارداد هوشمند دسترسی داشته باشند. این بایت‌کد‌ها در کاوشگرهای بلاکچین نیز قابل مشاهده هستند.
همچنین
\gls{Decompiler}هایی
وجود دارند که از بایت‌کد‌های قرارداد هوشمند کد سالیدیتی آن را به دست می‌آورند.
پس در نتیجه تلاش برای مخفی کردن کدهای قرارداد هوشمند بیهوده خواهد بود.
در تصویر
\ref{fig:cappu-bytecodes}
می‌توان بایت‌کدهای کاپو در اتراسکن را مشاهده کرد.
همچنین دکمه دیکامپایل کردن بایت‌کد‌ها نیز در تصویر مشخص است.

\begin{figure}[H]
\centerline{\frame{\includegraphics[width=14cm]{cappu-bytecodes.png}}}
\caption{بایت‌کدهای قرارداد کاپو}
\label{fig:cappu-bytecodes}
\end{figure}

دلیل دوم، اصلی‌ترین مزیت اپلیکیشن‌های غیرمتمرکز نسبت به اپلیکیشن‌های متمرکز عدم نیاز به اعتماد است.
کاربرها می‌توانند کد‌های قرارداد هوشمند را بخوانند و به کد نوشته شده اعتماد کنند.‌
در حالی که اگر کد برنامه برای همه کاربران قابل مشاهده نباشد کاربرها باید به سازندگان آن برنامه اعتماد کنند.

دلیل سوم، بارگذاری کردن قرارداد‌های هوشمند معمولا آسان نیست
و سرعت تغییرات پایین‌تر از اپلیکیشن‌های متمرکز هست.‌
پس امکان این که با پیدا شدن هر مشکل بتوان به سرعت آن را درست کرد کمتر وجود دارد
و مسئله امنیت بسیار با اهمیت می‌شود.
متن‌باز نوشتن قرارداد هوشمند باعث می‌شود چشم‌های بیشتری کدهای قرارداد را بخوانند
و مشکلات احتمالی سریعتر مشخص و رفع شوند.
تعداد زیادی از پروژه‌ها مانند کاپو از همان روز اول قرارداد هوشمند را به صورت متن‌باز توسعه می‌دهند،
بعضی نیز ترجیح میدهند که پروژه به مرحله‌ای از توسعه برسد و سپس آن را متن‌باز میکنند.

در این حوزه سرعت پیشرفت و توسعه به دلیل متن باز بودن به شدت بالاست
به نحوی که در طی اجرای این پروژه مرج ریکوئستی روی کتابخانه
\gls{OpenZeppelin}
زده شد که در همان روز مرج گردید.
این موضوع علاوه بر این که نشان‌دهنده سرعت پیشرفت بسیار بالاست،
این واقعیت را نیز نشان میدهد که در یک جامعه متن‌باز هر توسعه‌دهنده
می‌تواند به پیشرفت جامعه به هر شکلی که می‌پسندد کمک کند،
اشکالاتی که مشاهده می‌کند را گزارش دهد یا تصحیح کند.

برای مثال همکاری
\LTRfootnote{\url{https://github.com/OpenZeppelin/openzeppelin-contracts/pull/3314}}
نویسنده‌ی این پایان‌نامه در پروژه اپن‌زپلین را می‌توان در تصویر
\ref{fig:zeppelin-merge-req}
مشاهده کرد.
این مثال نشان‌دهنده این است که حتی در پروژه بسیار بزرگی مانند اپن‌زپلین نیز
از کمک عموم توسعه‌دهندگان به راحتی پذیرش می شود.

\begin{figure}[H]
\centerline{\frame{\includegraphics[width=14cm]{OpenZepplinContribution.png}}}
\caption{در طی انجام پروژه مرج ریکوئستی روی اپن‌زپلین باز شد که در همان روز مرج گردید.}
\label{fig:zeppelin-merge-req}
\end{figure}


% ------------ Section 2.2
\section{آشنایی با مفهوم توکن تعویض‌ناپذیر}
شروع رمزارزها با توکن‌های تعویض‌پذیر بود.
مفهوم تعویض‌پذیری به این معنی است که یک توکن با توکن دیگر تفاوتی ندارد
و با جابه‌جا شدن آن‌ها تغییری ایجاد نمی‌شود.
برای مثال یک بیت‌کوین با بیت‌کوین دیگر هیچ تفاوتی ندارد.

اما توکن‌های تعویض‌ناپذیر اینگونه نیستند،
هر یک از آن‌ها منحصر به فرد است و نمی‌توان آن‌ها را به جای یکدیگر به کار برد.
در دنیای واقعی خانه می‌تواند مثال خوبی از یک دارایی تعویض‌ناپذیر باشد.
هیچ دو خانه‌ای دقیقا شبیه به هم، در یک مکان، در طبقه یکسان و دارای پلاک مشترک نیستند.

پس مثلا به عنوان یک کاربرد،
شهرداری می‌تواند یک قرارداد هوشمند ایجاد کند
و به هر خانه یک توکن تعویض‌ناپذیر اختصاص دهد.
به این صورت صاحب خانه به جای سند یک توکن تعویض‌ناپذیر دارد
که مشخص می‌کند آن خانه متعلق به اوست.
فروش خانه به راحتی انتقال آن توکن تعویض‌ناپذیر به شخص دیگری خواهد بود.

از نظر فنی هر توکن به این صورت یکتاست که یک
\lr{Token ID}
یکتا در قراردادش دارد و هر قرارداد هم دارای یک آدرس یکتا در شبکه بلاکچین است. پس ترکیب
\lr{Contract Address}
و
\lr{Token ID}
باعث می‌شود که هر توکن در کل شبکه بلاکچین یکتا باشد.


% ------------ Section 2.3
\section{کاربردها، حال و آینده}
کاربرد توکن‌های تعویض‌ناپذیر تا به حال در دو دسته خلاصه می‌شود.
دسته اول به عنوان صاحب یک اثر دیجیتال،
مانند یک تصویر یا یک موسیقی.
.دسته دوم به عنوان یک جواز یا بلیت برای ورود به جایی یا دریافت کالایی
برای مثال همایشی برگزار می‌شود که فقط دارندگان توکن‌های یک قرارداد هوشمند می‌توانند به آن وارد شوند.

معروف‌ترین پلتفرم معاملاتی این توکن‌ها
\gls{OpenSea}
است که می‌توان در آن توکن‌های موجود را مشاهده کرد و یک توکن را توسط مزایده خرید و یا به فروش گذاشت.
اپن‌سی در حال حاضر از قراردادهای شبکه‌های اتریوم و سولانا پشتیبانی می‌کند.
دیگر شبکه‌ها نیز برای این منظور پلتفرم‌های خود را دارند، مانند شبکه
\gls{Atom}
که در آن از پلتفرم
\gls{Stargaze}
برای معامله توکن‌های تعویض‌ناپذیر استفاده می‌شود.

کاربردهای توکن‌های تعویض‌ناپذیر در آینده می‌تواند بسیار وسیع باشد.
دارایی‌های فیزیکی دنیای واقعی، بلیت‌های ورود به یک مکان یا یک همایش،
دارایی‌های دنیای مجازی مانند یک موسیقی یا دارایی در یک بازی
و حتی دامنه‌های اینترنتی همه می‌توانند به توکن‌های تعویض‌ناپذیر تبدیل شوند.
مزایای تبدیل این موارد به توکن‌های تعویض‌ناپذیر قابلیت نگهداری آسان‌تر،
فروش و انتقال راحت‌تر، امنیت بیشتر، آزادی در تراکنش‌ها
و آشکار بودن مالکیت دارایی بر همگان است.


% ------------ Section 2.4
\section{قرارداد‌های هوشمند و استانداردسازی}
اکثر قرارداد‌های هوشمند قابلیت‌هایی مشابه با یکدیگر دارند.
برای مثال گروهی از قرارداد‌های هوشمند توکن‌های تعویض‌پذیر دارند
و گروهی توکن‌های تعویض‌ناپذیر.
از طرفی اپلیکیشن‌هایی مانند کیف‌پول‌های دیجیتال،
پلتفرم‌های معاملاتی و صرافی‌ها
نیاز دارند که بتوانند دارایی‌های کاربر
اعم از توکن‌های تعویض‌پذیر و تعویض‌ناپذیر را ببینند،
به همین دلیل باید از نحوه صحبت کردن با قراردادهای هوشمند مطلع باشند.

برای ساده‌تر شدن این فرایند
و همسان‌سازی اینترفیس این قراردادهای هوشمند استانداردهایی تعریف شده است
که با استفاده از این استانداردها، هم فرایند توسعه قرارداد هوشمند آسان‌تر خواهد شد
و هم ارتباط میان قرارداد هوشمند و اپلیکیشن‌های دیگر مانند کیف‌پول‌ها، پلتفرم‌های معاملاتی و ... آسان‌تر برقرار خواهد شد.

از نمونه‌های معروف این استانداردها
\lr{ERC20}
برای قرارداد‌هایی با توکن‌های تعویض‌پذیر و
\lr{ERC721}
برای قراردادهایی با توکن‌های تعویض‌ناپذیر است. در این پروژه از استاندارد
\lr{ERC721}
استفاده می‌شود اما در مورد
\lr{ERC1155}
هم مطالعه شده و توضیح داده می‌شود.
به طور خلاصه
\lr{ERC1155}
قابلیت‌های بیشتری از
\lr{ERC721}
دارد.
قراردادی با این استاندارد می‌تواند هم توکن‌های تعویض‌پذیر و هم تعویض‌ناپذیر داشته باشد.
جزئیات بیشتر هر یک از این قراردادها را می‌توان در مستندات اتریوم
\cite{EthereumDocs}
مطالعه کرد.

برای استفاده از این استاندارد‌ها از بسته‌های متن‌بازی استفاده می‌شود که این استاندارد‌ها را پیاده‌سازی کرده‌اند
و از آن‌ها در قرارداد مورد نظر ارث‌بری می‌شود.
یکی از بهترین پیاده‌سازی‌های این استاندارد‌ها توسط اپن‌زپلین
\cite{ZeppelinDocs}
انجام شده است که در این پروژه نیز از همین پیاده‌سازی استفاده می‌شود.

\subsection{استاندارد \lr{ERC20}}
این استاندارد مناسب توکن‌های تعویض‌پذیر است.
اینترفیسی تعریف می‌کند که نیازهای قراردادهایی با توکن‌های تعویض‌پذیر برطرف شود
و نحوه تعامل برقرار کردن با آن‌ها یکسان گردد.
در این استاندارد فقط می‌توان یک نوع توکن تعویض‌پذیر به تعداد دلخواه داشت.
این استاندارد متدهایی برای تعریف حداکثر تعداد توکن‌های موجود،
گرفتن موجودی یک آدرس، و انتقال توکن‌ها دارد. توضیحات دقیق‌تر در مورد این استاندارد را می‌توان در وبسایت
اتریوم
\LTRfootnote{\url{https://ethereum.org/en/developers/docs/standards/tokens/erc-20}}
یا اپن‌زپلین
\LTRfootnote{\url{https://docs.openzeppelin.com/contracts/4.x/api/token/erc20}}
مشاهده کرد.

\subsection{استاندارد \lr{ERC721}}
استفاده از استاندارد
\lr{ERC721}
برای توکن‌های تعویض‌ناپذیر بسیار مرسوم است.
در این استاندارد متدها و ایونت‌هایی برای یکسان سازی اینترفیس قراردادهای دارای توکن‌های تعویض‌ناپذیر تعریف شده است.
در این نوع قرارداد‌ها می‌توان به تعداد دلخواه توکن‌های متفاوت با یکدیگر داشت،
هر توکن یک شناسه یکتا دارد که می‌تواند به صورت ترتیبی یا غیر ترتیبی ایجاد شود.

همچنین متدی وجود دارد که می‌تواند شناسه یک توکن را به آدرسی تبدیل کند که اطلاعات آن توکن در آنجا موجود است.
کاربرها می‌توانند توکن‌هایی که دارند را مشاهده کنند،
به یکدیگر ارسال کنند یا به آدرس دیگری وکالت بدهند که توکن‌ها را به شخص دیگری ارسال کند.

تنها قابلیتی که به طور مشخص در این قرارداد معین نشده است که چگونه باید انجام شود،
قابلیت ساخت توکن‌ها است.
اکثر قراردادهای هوشمندی که توکن‌های تعویض‌ناپذیر دارند به کاربران اجازه ساخت توکن‌ها را نمی‌دهند
و ساخت توکن‌ها فقط به آدرس صاحب قرارداد محدود می‌شود.
اما در کاپو اینگونه نیست و هرکسی می‌تواند برای خودش توکن بسازد.

اطلاعات دقیق‌تر در مورد این استاندارد را نیز می‌توان در وبسایت
اتریوم
\LTRfootnote{\url{https://ethereum.org/en/developers/docs/standards/tokens/erc-721}}
یا
اپن‌زپلین
\LTRfootnote{\url{https://docs.openzeppelin.com/contracts/4.x/api/token/erc721}}
مشاهده کرد.


\subsection{استاندارد \lr{ERC1155}}
تا اینجا با معروف‌ترین استاندارد‌های موجود برای قراردادهایی که توکن‌های تعویض‌پذیر یا تعویض‌ناپذیر دارند آشنا شدیم.
اما همچنان نیازمندی‌هایی وجود دارند که توسط هیچ‌یک از این استانداردها برطرف نمی‌شوند. نیازمندی‌هایی مانند:
\begin{itemize}
	\item
داشتن توکن‌های تعویض‌ناپذیر با تعداد محدود به جای فقط یکی.
	\item
داشتن همزمان چندین نوع توکن مختلف در یک قرارداد.
	\item
انتقال همزمان چند توکن از انواع مختلف از کاربری به کاربر دیگر.
\end{itemize}

یک مثال از کاربردی که به این قابلیت‌ها نیاز دارد می‌تواند یک بازی مثل
\gls{Monopoly}
باشد
که در آن هر کاربر مقداری پول دارد که در واقع یک توکن تعویض‌پذیر هست،
به عنوان دارایی چند خانه دارد که به عنوان توکن‌های تعویض‌ناپذیری هستند که از هرکدام فقط یکی وجود دارد
و ممکن است چند کارت خروج از زندان داشته باشد که یکتا نیستند اما تعداد محدودی در بازی وجود دارد. استاندارد
\lr{ERC1155}
همه‌ی این نیازها را برطرف می‌کند. همه‌ی این چند نوع توکن می‌توانند همزمان در یک قرارداد هوشمند وجود داشته باشند.

در این استاندارد متدهایی برای تعریف نوعی توکن با تعداد مشخص وجود دارد.
اگر نیاز به توکنی تعویض‌ناپذیر باشد تعداد آن یک قرارداده می‌شود.
همچنین متدهایی برای ارسال تعداد مشخص از چند نوع توکن مختلف در یک تراکنش،
دادن وکالت توکن‌ها به آدرس دیگر و گرفتن موجودی یک آدرس در این استاندارد وجود دارد.

اطلاعات دقیق‌تر در مورد این استاندارد را نیز می‌توان در وبسایت
اتریوم
\LTRfootnote{\url{https://ethereum.org/en/developers/docs/standards/tokens/erc-1155}}
یا
اپن‌زپلین
\LTRfootnote{\url{https://docs.openzeppelin.com/contracts/4.x/api/token/erc1155}}
مشاهده کرد.
